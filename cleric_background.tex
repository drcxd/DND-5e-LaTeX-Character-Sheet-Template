\documentclass[letterpaper,twocolumn,openany,nodeprecatedcode]{dndbook}

% Use babel or polyglossia to automatically redefine macros for terms
% Armor Class, Level, etc...
% Default output is in English; captions are located in lib/dndstrings.sty.
% If no captions exist for a language, English will be used.
%1. To load a language with babel:
%	\usepackage[<lang>]{babel}
%2. To load a language with polyglossia:
%	\usepackage{polyglossia}
%	\setdefaultlanguage{<lang>}
%% \usepackage[english]{babel}
\usepackage{ctex}
%\usepackage[italian]{babel}
% For further options (multilanguage documents, hypenations, language environments...)
% please refer to babel/polyglossia's documentation.

\usepackage[utf8]{inputenc}
\usepackage[singlelinecheck=false]{caption}
%% \usepackage{lipsum}
%% \usepackage{listings}
%% \usepackage{shortvrb}
%% \usepackage{stfloats}

%% \captionsetup[table]{labelformat=empty,font={sf,sc,bf,},skip=0pt}

%% \MakeShortVerb{|}

%% \lstset{%
%%   basicstyle=\ttfamily,
%%   language=[LaTeX]{TeX},
%%   breaklines=true,
%% }

%% \title{The Dark \LaTeX{} \\
%% \large An Example of the dndbook Class}
%% \author{The rpgTeX Team}
%% \date{2020/04/21}

\begin{document}

%% \frontmatter

%% \maketitle

%% \tableofcontents

\mainmatter%

\section{矮人}

\footnotesize

矮人的王国古老而壮丽,厅堂建在山脉深处,镐与锤在矿井和锻炉旁叮铛作响。每个矮人都继承了对家族和传统的忠诚,以及对哥布林和兽人的憎恶。

勇敢而坚强,矮人们作为熟练的战士,矿工和金石工匠而广为人知。矮人的寿命往往可达400年以上,所以最年长的矮人仿佛来自另一个世界。

矮人们像他们所热爱的高山一样坚忍不拔,渡过了漫长的岁月。每个矮人都坚定而忠诚,言出必行,有时甚至过于固执。

在遗忘国度(the Forgotten Realms), 你的人民被称作黄金矮人。他们在遥远的南方建立王国,并和人类的事务保持距离。在北方有他们的兄弟,盾牌矮人,他们强壮,坚韧,习惯于在恶劣的土地上生活。多年以前,你前往了北方国度。

\section{牧师}

牧师是世俗和遥远神界的中间人。尽管信奉不同的神明,牧师们都致力于将自己神明的神迹带到世间。不像普通的神父或教堂的侍者,牧师具有神圣的魔力。

\textbf{神圣领域。}神圣领域是一个具有魔法效应的领域,你从那里汲取魔力,这个领域和你侍奉的神祇息息相关。你的领域使你可以一直准备某些魔法,例如 bless 和 cure wouds.

你的领域是生命,和很多善良的神祇密切相关。你的神明,Marthammor Duin, 是旅行者,流浪者和被放逐者的矮人神明,保祐他们平安穿行于陌生的土地和人群。信奉他的矮人在银或铁制的项链上配带着他的圣符,其形象是一只靴子覆盖在一把权杖上。

\section{背景故事}

在 Mintarn 的岛屿上被训练为一名士兵后,你和佣兵团一起前往 Neverwinter, 在那里负责保卫和治安。你逐渐厌倦了你的同袍,相比于保护平民,他们更喜欢在平民头上作威作福。这一切在最近让你爆发了,你拒绝了命令,而是选择了遵从自己的良知。你被禁止继续服役,尽管你保留了你的军衔和与雇佣兵的联系。在这之后,你全身心地投入到侍奉自己的神明上。

\textbf{个人目标:教训 Readbrands。}你听说 Pahndalin 镇的 Daran Edermath 在找有勇气和原则的人来教训一些当地恶徒。这些 Readbrands 的恶徒在 Phandalin 无法无天,就像你曾经的战友在 Neverwinter 一样。停止他们的恶行是一件有意义的事情。

\textbf{阵营:中立善良。}你的良知,而不是法律和权威,指引你做正确的事情。力量应该被用于造福全体,而不是压迫弱者。

\section{等级提升}

如规则书所述,在你冒险和克服挑战的过程中,你会获得经验值(XP)。

每当你提升等级时,你获得一个额外的 Hit Die 并将 1d8 + 3 加到你的最大生命值上。

当你升级时,你掌握更多的法术。你可以准备数量等于 你的等级+你的 Wisdom 修正值 的不同法术,如施法提升表所示。你还能获得更多的法术位。

\begin{DndTable}[header=施法提升]{c c c c c}
       &             & \multicolumn{3}{c}{各法术等级的法术位数量} \\
  等级 & 准备法术数量 & 1级 & 2级 & 3级 \\
  2    & 5           &  3  & --  & --  \\
  3    & 6           &  4  & 2   & --  \\
  4    & 8           &  4  & 3   & --  \\
  5    & 9           &  4  & 3   & 2
\end{DndTable}

\subsubsection{2级:300XP}

\textbf{引导神力(Channel Divinity)。}你可以从你的神明处直接引导神圣能量,用来使用以下两个魔法效果之一: Turn Undead 或 Preserve Life. 任一个效果需要消耗你的动作,并需要圣符。当你使用引导神力时,从两个效果中选择一个。你必须在短休或长休后才能再次使用引导神力。

\textbf{驱散不死生物(Turn Undead)。}当你使用驱散不死生物时,每个距你30英尺以内,可以看见或听见你的不死生物必须通过一个智力豁免(豁免难度13)。如果一个生物没能豁免,则它被驱散1分钟或直到它受到任何伤害。

一个被驱散的生物必须在它的回合尝试尽可能远离你,并且它不能主动移动到距你30英尺以内。它不能使用反应。它的动作只能用来使用冲刺,或尝试摆脱使其不能移动的状态。如果它没有可以移动的地方,该生物可以使用闪避动作。

\textbf{守护生命(Preserve Life)。}当你使用守护生命时,选择治疗距你30英尺以内的1个或多个生物共10点生命值。该特性最多治疗一个生物其最大生命值的一半。在3级,你可以共治疗15点生命值,4级20点,5级25点。

\subsubsection{3级:900XP}
\textbf{法术(Spells)。}你现在可以准备并施放2级法术。除了你选择准备的法术外,你还能一直准备两个额外的领域法术: lesser restoration 和 spiritual weapon.

\subsubsection{4级:2700XP}
\textbf{法术(Spells)。}你学习一个新的牧师戏法。
\textbf{能力值提升(Ability Score Improvement)。}你的 Wisdom 增加到18,并带来如下影响:
\begin{itemize}
\item 你的 Wisdom 修正值变为+4.
\item 你的法术豁免等级和你的驱散不死生物的豁免等级各增加1.
\item 你的法术攻击加成增加1.
\item 你的 Wisdom 豁免修正值增加1.
\item 你的 Wisdom 技能修正值增加1.
\item 因为你的 Perception 技能的修正值增加,你的被动 Wisdom (Perception) 数值增加1.
\end{itemize}

\subsubsection{5级:6500XP}
\textbf{法术(Spells)。}你现在可以准备并施放3级法术。除了你选择准备的法术,你还能一直准备两个领域法术: beacon of hope 和 revivify.
\textbf{熟练项加成(Proficiency Bonus)。}你的熟练项加成增加至+3,并带来如下影响:
\begin{itemize}
\item 你的法术攻击加成和你所熟练武器的攻击加成增加1.
\item 你的法术豁免等级和你的驱散不死生物的豁免等级各增加1.
\item 你所熟练的豁免和技能的修正值增加1.
\end{itemize}

\textbf{摧毁不死生物(Destroy Undead)。}当一个不死生物没能通过你的驱散不死生物的豁免时,如果该生物的挑战评级小于等于1/2,则该生物马上被摧毁。

\subsubsection{改善你的护甲}
在你获得财宝后,你可以购买更好的护甲来提高你的护甲等级。规则书中包含装备,包括护甲,的更多信息。

%% command to compile this file: latexmk --interaction=nonstopmode --pdf --pdflatex=pdflatex test.tex

%% \part{Layout}

%% \chapter{Sections}

%% \DndDropCapLine{T}{his package is designed to aid you in} writing beautifully typeset documents for the fifth edition of the world's greatest roleplaying game. It starts by adjusting the section formatting from the defaults in \LaTeX{} to something a bit more familiar to the reader. The chapter formatting is displayed above.

%% \section{Section}
%% Sections break up chapters into large groups of associated text.

%% \subsection{Subsection}
%% Subsections further break down the information for the reader.

%% \subsubsection{Subsubsection}
%% Subsubsections are the furthest division of text that still have a block header. Below this level, headers are displayed inline.

%% \paragraph{Paragraph}
%% The paragraph format is seldom used in the core books, but is available if you prefer it to the ``normal'' style.

%% \subparagraph{Subparagraph}
%% The subparagraph format with the paragraph indent is likely going to be more familiar to the reader.

%% \section{Special Sections}
%% The module also includes functions to aid in the proper typesetting of multi-line section headers: |\DndFeatHeader| for feats, |\DndItemHeader| magic items and traps, and |\DndSpellHeader| for spells.

%% \DndFeatHeader{Typesetting Savant}[Prerequisite: \LaTeX{} distribution]
%% You have acquired a package which aids in typesetting source material for one of your favorite games, giving you the following benefits:

%% \begin{itemize}
%%   \item You have advantage on Intelligence checks to typeset new content.
%%   \item When you fail an Intelligence check to typeset new content, you can ask questions online at the package's website.
%% \end{itemize}

%% \DndItemHeader{Foo's Quill}{Wondrous item, rare}
%% This quill has 3 charges. While holding it, you can use an action to expend 1 of its charges. The quill leaps from your hand and writes a contract applicable to your situation.

%% The quill regains 1d3 expended charges daily at dawn.

%% \DndSpellHeader%
%%   {Beautiful Typesetting}
%%   {4th-level illusion}
%%   {1 action}
%%   {5 feet}
%%   {S, M (ink and parchment, which the spell consumes)}
%%   {Until dispelled}
%% You are able to transform a written message of any length into a beautiful scroll. All creatures within range that can see the scroll must make a wisdom saving throw or be charmed by you until the spell ends.

%% While the creature is charmed by you, they cannot take their eyes off the scroll and cannot willingly move away from the scroll. Also, the targets can make a wisdom saving throw at the end of each of their turns. On a success, they are no longer charmed.

%% \section{Map Regions}
%% The map region functions |\DndArea| and |\DndSubArea| provide automatic numbering of areas.

%% \DndArea{Village of Hommlet}
%% This is the village of hommlet.

%% \DndSubArea{Inn of the Welcome Wench}
%% Inside the village is the inn of the Welcome Wench.

%% \DndSubArea{Blacksmith's Forge}
%% There's a blacksmith in town, too.

%% \DndArea{Foo's Castle}
%% This is foo's home, a hovel of mud and sticks.

%% \DndSubArea{Moat}
%% This ditch has a board spanning it.

%% \DndSubArea{Entrance}
%% A five-foot hole reveals the dirt floor illuminated by a hole in the roof.

%% \chapter{Text Boxes}

%% The module has three environments for setting text apart so that it is drawn to the reader's attention. |DndReadAloud| is used for text that a game master would read aloud.

%% \begin{DndReadAloud}
%%   As you approach this module you get a sense that the blood and tears of many generations went into its making. A warm feeling welcomes you as you type your first words.
%% \end{DndReadAloud}

%% \section{As an Aside}
%% The other two environments are the |DndComment| and the |DndSidebar|. The |DndComment| is breakable and can safely be used inline in the text.

%% \begin{DndComment}{This Is a Comment Box!}
%%   A |DndComment| is a box for minimal highlighting of text. It lacks the ornamentation of |DndSidebar|, but it can handle being broken over a column.
%% \end{DndComment}

%% The |DndSidebar| is not breakable and is best used floated toward a page corner as it is below.

%% \begin{DndSidebar}[float=!b]{Behold the DndSidebar!}
%%   The |DndSidebar| is used as a sidebar. It does not break over columns and is best used with a figure environment to float it to one corner of the page where the surrounding text can then flow around it.
%% \end{DndSidebar}

%% \section{Tables}
%% The |DndTable| colors the even rows and is set to the width of a line by default.

%% \begin{DndTable}[header=Nice Table]{XX}
%%     Table head  & Table head \\
%%     Some value  & Some value \\
%%     Some value  & Some value \\
%%     Some value  & Some value
%% \end{DndTable}

%% \chapter{Monsters and NPCs}

%% % Monster stat block
%% \begin{DndMonster}[float*=b,width=\textwidth + 8pt]{Monster Foo}
%%   \begin{multicols}{2}
%%     \DndMonsterType{Medium aberration (metasyntactic variable), neutral evil}

%%     % If you want to use commas in the key values, enclose the values in braces.
%%     \DndMonsterBasics[
%%         armor-class = {9 (12 with \emph{mage armor})},
%%         hit-points  = {\DndDice{3d8 + 3}},
%%         speed       = {30 ft., fly 30 ft.},
%%       ]

%%     \DndMonsterAbilityScores[
%%         str = 12,
%%         dex = 8,
%%         con = 13,
%%         int = 10,
%%         wis = 14,
%%         cha = 15,
%%       ]

%%     \DndMonsterDetails[
%%         %saving-throws = {Str +0, Dex +0, Con +0, Int +0, Wis +0, Cha +0},
%%         %skills = {Acrobatics +0, Animal Handling +0, Arcana +0, Athletics +0, Deception +0, History +0, Insight +0, Intimidation +0, Investigation +0, Medicine +0, Nature +0, Perception +0, Performance +0, Persuasion +0, Religion +0, Sleight of Hand +0, Stealth +0, Survival +0},
%%         %damage-vulnerabilities = {cold},
%%         %damage-resistances = {bludgeoning, piercing, and slashing from nonmagical attacks},
%%         %damage-immunities = {poison},
%%         %condition-immunities = {poisoned},
%%         senses = {darkvision 60 ft., passive Perception 10},
%%         languages = {Common, Goblin, Undercommon},
%%         challenge = 1,
%%       ]
%%     % Traits
%%     \DndMonsterAction{Innate Spellcasting}
%%     Foo's spellcasting ability is Charisma (spell save DC 12, +4 to hit with spell attacks). It can innately cast the following spells, requiring no material components:
%%     \begin{DndMonsterSpells}
%%       \DndInnateSpellLevel{misty step}
%%       \DndInnateSpellLevel[3]{fog cloud, rope trick}
%%       \DndInnateSpellLevel[1]{identify}
%%     \end{DndMonsterSpells}

%%     \DndMonsterAction{Spellcasting}
%%     Foo is a 2nd-level spellcaster. Its spellcasting ability is Charisma (spell save DC 12, +4 to hit with spell attacks). It has the following sorcerer spells prepared:
%%     \begin{DndMonsterSpells}
%%       \DndMonsterSpellLevel{blade ward, fire bolt, light, shocking grasp}
%%       \DndMonsterSpellLevel[1][3]{burning hands, mage armor, shield}
%%     \end{DndMonsterSpells}

%%     \DndMonsterSection{Actions}
%%     \DndMonsterAction{Multiattack}
%%     The foo makes two melee attacks.

%%     %Default values are shown commented out
%%     \DndMonsterAttack[
%%       name=Dagger,
%%       %distance=both, % valid options are in the set {both,melee,ranged},
%%       %type=weapon, %valid options are in the set {weapon,spell}
%%       mod=+3,
%%       %reach=5,
%%       %range=20/60,
%%       %targets=one target,
%%       dmg=\DndDice{1d4+1},
%%       dmg-type=piercing,
%%       %plus-dmg=,
%%       %plus-dmg-type=,
%%       %or-dmg=,
%%       %or-dmg-when=,
%%       %extra=,
%%     ]

%%     %\DndMonsterMelee calls \DndMonsterAttack with the melee option
%%     \DndMonsterMelee[
%%       name=Flame Tongue Longsword,
%%       mod=+3,
%%       %reach=5,
%%       %targets=one target,
%%       dmg=\DndDice{1d8+1},
%%       dmg-type=slashing,
%%       plus-dmg=\DndDice{2d6},
%%       plus-dmg-type=fire,
%%       or-dmg=\DndDice{1d10+1},
%%       or-dmg-when=if used with two hands,
%%       %extra=,
%%     ]

%%     %\DndMonsterRanged calls \DndMonsterAttack with the ranged option
%%     \DndMonsterRanged[
%%       name=Assassin's Light Crossbow,
%%       mod=+1,
%%       range=80/320,
%%       dmg=\DndDice{1d8},
%%       dmg-type=piercing,
%%       %plus-dmg=,
%%       %plus-dmg-type=,
%%       %or-dmg=,
%%       %or-dmg-when=,
%%       extra={, and the target must make a DC 15 Constitution saving throw, taking 24 (7d6) poison damage on a failed save, or half as much damage on a successful one}
%%     ]

%%     % Legendary Actions
%%     \DndMonsterSection{Legendary Actions}
%%     The foo can take 3 legendary actions, choosing from the options below. Only one legendary action option can be used at a time and only at the end of another creature's turn. The foo regains spent legendary actions at the start of its turn.

%%     \begin{DndMonsterLegendaryActions}
%%       \DndMonsterLegendaryAction{Move}{The foo moves up to its speed.}
%%       \DndMonsterLegendaryAction{Dagger Attack}{The foo makes a dagger attack.}
%%       \DndMonsterLegendaryAction{Create Contract (Costs 3 Actions)}{The foo presents a contract in a language it knows and waves it in the face of a creature within 10 feet. The creature must make a DC 10 Intelligence saving throw. On a failure, the creature is incapacitated until the start of the foo's next turn. A creature who cannot read the language in which the contract is written has advantage on this saving throw.}
%%     \end{DndMonsterLegendaryActions}
%%   \end{multicols}
%% \end{DndMonster}

%% The |DndMonster| environment is used to typeset monster and NPC stat blocks. The module supplies many functions to easily typeset the contents of the stat block

%% \part{Customization}

%% \chapter{Colors}

%% \begin{table*}[b]
%%   \caption{\DndFontTableTitle{}Colors Supported by this Package}\label{tab:colors}

%%   \begin{tabularx}{\linewidth}{lX}
%%     \textbf{Color}                  & \textbf{Description} \\
%%     \rowcolor{PhbLightGreen}
%%     |PhbLightGreen|                 & Light green used in PHB Part 1 (Default) \\
%%     \rowcolor{PhbLightCyan}
%%     |PhbLightCyan|                  & Light cyan used in PHB Part 2 \\
%%     \rowcolor{PhbMauve}
%%     |PhbMauve|                      & Pale purple used in PHB Part 3 \\
%%     \rowcolor{PhbTan}
%%     |PhbTan|                        & Light brown used in PHB appendix \\
%%     \rowcolor{DmgLavender}
%%     |DmgLavender|                   & Pale purple used in DMG Part 1 \\
%%     \rowcolor{DmgCoral}
%%     |DmgCoral|                      & Orange-pink used in DMG Part 2 \\
%%     \rowcolor{DmgSlateGray}
%%     |DmgSlateGray| (|DmgSlateGrey|) & Blue-gray used in PHB Part 3 \\
%%     \rowcolor{DmgLilac}
%%     |DmgLilac|                      & Purple-gray used in DMG appendix \\
%%     \rowcolor{BrGreen}
%%     |BrGreen|                       & Gray-green used for tables in Basic Rules\\
%%   \end{tabularx}
%% \end{table*}

%% This package provides several global color variables to style |DndComment|, |DndReadAloud|, |DndSidebar|, and |DndTable| environments.

%% \begin{DndTable}[header=Box Colors]{lX}
%%   Color            &  Description \\
%%   |commentcolor|   & |DndComment| background \\
%%   |readaloudcolor| & |DndReadAloud| background \\
%%   |sidebarcolor|   & |DndSidebar| background \\
%%   |tablecolor|     & background of even |DndTable| rows \\
%% \end{DndTable}

%% They also accept an optional color argument to set the color for a single instance. See Table~\ref{tab:colors} for a list of core book accent colors.

%% \begin{lstlisting}
%% \begin{DndTable}[color=PhbLightCyan]{cX}
%%   d8 & Item \\
%%   1  & Small wooden button \\
%%   2  & Red feather \\
%%   3  & Human tooth \\
%%   4  & Vial of green liquid \\
%%   5  & Loaded dice \\
%%   6  & Tasty biscuit \\
%%   7  & Broken axe handle \\
%%   8  & Tarnished silver locket \\
%% \end{DndTable}
%% \end{lstlisting}

%% \begin{DndTable}[color=PhbLightCyan]{cX}
%%   d8 & Item \\
%%   1  & Small wooden button \\
%%   2  & Red feather \\
%%   3  & Human tooth \\
%%   4  & Vial of green liquid \\
%%   5  & Loaded dice \\
%%   6  & Tasty biscuit \\
%%   7  & Broken axe handle \\
%%   8  & Tarnished silver locket \\
%% \end{DndTable}

%% \section{Themed Colors}
%% Use |\DndSetThemeColor[<color>]| to set |commentcolor|, |readaloudcolor|, |sidebarcolor|, and |tablecolor| to a specific color. Calling |\DndSetThemeColor| without an argument sets those colors to the current |themecolor|. In the following example the group limits the change to just a few boxes; after the group finishes, the colors are reverted to what they were before the group started.

%% \begin{lstlisting}
%% \begingroup
%% \DndSetThemeColor[PhbMauve]

%% \begin{DndComment}{This Comment Is in Mauve}
%%   This comment is in the the new color.
%% \end{DndComment}

%% \begin{DndSidebar}{This Sidebar Is Also Mauve}
%%   The sidebar is also using the new theme color.
%% \end{DndSidebar}
%% \endgroup
%% \end{lstlisting}

%% \begingroup
%% \DndSetThemeColor[PhbMauve]

%% \begin{DndComment}{This Comment Is in Mauve}
%%   This comment is in the the new color.
%% \end{DndComment}

%% \begin{DndSidebar}{This Sidebar Is Also Mauve}
%%   The sidebar is also using the new theme color.
%% \end{DndSidebar}
%% \endgroup

\end{document}
